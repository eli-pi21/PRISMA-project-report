Si è reso necessario fornire un'interfaccia accessibile tramite web browser per la modifica delle configurazioni e la gestione dei software operanti sui singoli nodi, nonché per rendere più veloce e pratica la consultazione dei dati raccolti. Si è voluto quindi elaborare le immagini e i file per permettere all'utente di interpretare in modo immediato e con facilità i dati raccolti dal nodo.

Queste richieste si sono tradotte con l'implementazione di una web application accessibile tramite VPN (cfr. sezione \ref{rete-VPN}) utilizzando come URL l'indirizzo IP del nodo. L'oggetto del tirocinio è stato quindi lo sviluppo di questa web app sia backend sia frontend.