Fireball e bolide sono termini astronomici per indicare meteore particolarmente brillanti e spettacolari che possono essere agevolmente viste anche di giorno da un'ampia regione. 

Per meteoroide si intende un frammento di asteroide o cometa in orbita attorno al Sole che ha una dimensione inferiore al metro. Le meteore, anche chiamate stelle cadenti, sono la traccia visibile dei meteoroidi che entrano nell'atmosfera terrestre con un'alta velocità. 

Un fireball è una meteora che raggiunge una luminosità uguale o superiore a quella di Venere, il terzo astro più brillante nel cielo. I fireball che esplodono e si frammentano durante la caduta sono chiamati in gergo tecnico bolidi, anche se i due termini sono spesso utilizzati indifferentemente. Durante la fase di ingresso in atmosfera, l'oggetto impattante è rallentato e riscaldato per attrito. Nella parte frontale il gas atmosferico è compresso e scaldato e forma una zona di shock. Parte dell'energia generata dall'attrito provoca l'erosione dell'oggetto, e nella maggior parte dei casi la sua successiva rottura. La frammentazione aumenta l'effetto dell'attrito, causando ulteriore erosione e frammentazione, fino a quando la differenza tra le forze di pressione di fronte e dietro l'oggetto ne provocano la completa e catastrofica distruzione. Sebbene in genere gli oggetti che generano un fireball non siano grandi a sufficienza per sopravvivere intatti al passaggio in atmosfera, spesso frammenti o meteoriti possono venir recuperati a terra.
\cite{bolide-fireball}