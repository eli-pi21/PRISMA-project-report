\begin{wrapfigure}{l}{0.3\textwidth}
    \vspace{-30pt}
    \begin{center}
    \includegraphics[width=0.3\textwidth]{images/logo.png}
    \end{center}
    \vspace{-22pt}
\end{wrapfigure}

Il progetto PRISMA prevede la realizzazione di una rete italiana di camere all-sky per l'osservazione di meteore brillanti (fireball e bolidi), al fine di determinare le orbite degli oggetti che le provocano e delimitare con un buon grado di approssimazione le aree dell'eventuale caduta di frammenti per poter recuperare le meteoriti.

Il monitoraggio sistematico della copertura nuvolosa e dell'attività elettrica sarà usato per la validazione di modelli meteorologici. I dati raccolti in maniera sistematica contribuiranno al perfezionamento dei modelli di interazione dei corpi cosmici con l'atmosfera che a tutt'oggi presentano ancora molte lacune a causa della mancanza di dati osservativi di qualità.
\cite{progetto-PRISMA}
