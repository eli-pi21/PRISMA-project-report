Per l'ideazione, lo sviluppo e il testing della web application si è fatto uso dei seguenti strumenti:

\begin{itemize}
    
\item
\textbf{Confluence} \cite{confluence}: workspace per il team dove si possono realizzare pagine dinamiche per creare, catturare e collaborare su qualsiasi idea e progetto.
Il suo impiego è stato destinato alla consultazione e alla stesura della documentazione relativa al progetto PRISMA.

\item
\textbf{NetBeans} \cite{netbeans}: ambiente di sviluppo integrato (IDE) multi-linguaggio. 
È stato utilizzato come IDE per l'implementazione della web application.

\item
\textbf{Oracle VirtualBox} \cite{oracle-vbox}: potente prodotto di virtualizzazione x86 e AMD64 / Intel64.
Con questo strumento si è virtualizzato il nodo di testing illustrato nella sezione \ref{conf-nodo-testing}.

\item
\textbf{Git} \cite{git}: sistema di controllo di versione gratis e open source.

\item
\textbf{GitHub} \cite{github}: servizio di hosting per progetti software. Il lavoro svolto e l'infrastruttura precedentemente realizzata (cfr. sezione \ref{infrastruttura-PRISMA}) è ospitata da questo servizio.

\item
\textbf{Docker Hub} \cite{docker-hub}: servizio di hosting fornito da Docker per trovare e condividere immagini. Le immagini dei container ospitati sul nodo sono qui disponibili.

\item
\textbf{Chrome DevTools} \cite{chrome-devtools}: insieme di strumenti per sviluppatori web integrati direttamente nel browser di Google Chrome. 
Il debug della web application è avvenuto mediante questi strumenti.

\end{itemize}