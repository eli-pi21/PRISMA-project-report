La web application che è stata illustrata in questa tesi è funzionante e operativa (si può consultare il codice su GitHub \cite{prisma-node-webmin}). Durante il tirocinio sono stati configurati due nodi in produzione, Codogno e Savelli, attualmente in funzione: il nodo di Codogno ha infatti rilevato il 22 giugno 2022 una meteora, ben visibile nella figura \ref{fig:codogno-2206} consultabile direttamente dalla web app.

La ricerca che si sta sviluppando attorno al progetto PRISMA è consistente e promettente, coinvolgendo numerosi appassionati e studenti da tutta Italia. Per mezzo di tutti questi contributi il progetto è funzionante e sta portando ottimi risultati: grazie ad esso oltre ad essere state individuate sono state già raccolte alcune meteore anche di piccole dimensioni, che altrimenti sarebbero passate inosservate.